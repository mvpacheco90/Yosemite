%\documentclass[review]{elsarticle}
\documentclass[5p]{elsarticle}


\usepackage{lineno,hyperref}
\usepackage[separate-uncertainty=true,multi-part-units=single]{siunitx}
\usepackage{color}
\modulolinenumbers[5]
\usepackage{mhchem}
\usepackage{enumitem}
%\usepackage{trackchanges}

\journal{Geomorphology}


\newcommand{\COMON}{\begin{color}{blue}}
\newcommand{\COMOFF}{\end{color}}

\newcommand{\alon}{\begin{color}{red}}
\newcommand{\aloff}{\end{color}}

%% APA style
\bibliographystyle{model5-names}\biboptions{authoryear}

%\addeditor{Alain}

%%%%%%%
% The trackchanges package adds five new LaTeX commands:
%
%  \note[editor]{The note}
%  \annote[editor]{Text to annotate}{The note}
%  \add[editor]{Text to add}
%  \remove[editor]{Text to remove}
%  \change[editor]{Text to remove}{Text to add}
%
% complete documentation is here: http://trackchanges.sourceforge.net/
%%%%%%%



\begin{document}


	\begin{frontmatter}

\title{Insights into the Paleo-elevation of Yosemite Valley Shortly after the Last Glacial Maximum.\\ \vspace{1cm}Insights into the Paleo-elevation of Yosemite Valley 15,000 years B.P. from the Royal Arches Meadow Rock Avalanche.\\ \vspace{1cm}Suggestions?\\ \vspace{1m}  Yosemite Valley Paleo-elevation  after the Last Glacial Maximum from Geophysical Data}

%% Group authors per affiliation:
\author[Marcus]{Marcus Pacheco\corref{cor1}}
\address[Marcus]{California State University, Fresno}
\cortext[cor1]{Corresponding author.}
\ead{mvpacheco90@mail.fresnostate.edu}

\author[Alain]{Alain Plattner}
\address[Alain]{University of Alabama}

\author[Chris]{[Christopher Pluhar}
\address[Chris]{California State University, Fresno}

\author[Greg]{Greg Stock}
\address[Greg]{Yosemite National Park}



										\begin{abstract}
										
\COMON This is tjust a Test abstract!\COMOFF Since the retreat of the Last Glacial Maximum (~15,000 years ago), rockfalls have been the major force shaping Yosemite Valley, California. Rock avalanches are an especially large rockfall/rock slide that extends far beyond the cliff where they originate. These events are infrequent, but can reach hundreds of meters into the valley, and deposit an extremely large volume of debris when compared with regular rockfall events. Yosemite Valley is home of at least ten rock avalanche deposits, with the Royal Arches Meadow rock avalanche (RAMRA), situated in eastern Yosemite Valley, being the oldest event (~14,000 yr BP). Because this event occurred shortly after the Last Glacial Maximum (LGM), mapping the interface between this rock avalanche and the underlying valley sediments can give us insights into the valley elevation and overall geomorphic state of Yosemite Valley shortly after the LGM.  Holocene aggradation covers parts of the deposit, reducing its surface expression. This represents a challenge for estimating the dimensions of the deposit. To overcome this obstacle, we used a combination of geophysical methods (Electrical Resistivity Tomography (ERT) and Ground Penetrating Radar (GPR)) to image the interface between the RAMRA and underlying valley sediments. The strong dielectric permittivity and electrical resistivity contrast between the rock avalanche and the underlying sediments make both electrical resistivity tomography and ground penetrating radar ideal methods for our purpose. This then allowed us to infer that the surface of the valley underneath the Royal Arches Meadow Rock Avalanche is in average 1209m with a variation of +/- 3.2m.

									\end{abstract}

					\begin{keyword}
GPR \sep ERT \sep Yosemite
%\texttt{elsarticle.cls}\sep \LaTeX\sep Elsevier \sep template
%\MSC[2010] 00-01\sep  99-00
					\end{keyword}

	\end{frontmatter}

%\linenumbers

\section{Introduction}

%\alon Marcus: Right now you have a couple of paragraphs / topics that are not in the right place. For example, in the ``study area'' section, you talk about which methods you will use and why. This should rather go into introduction, or methods, but not study area.  \href{https://www.elsevier.com/connect/11-steps-to-structuring-a-science-paper-editors-will-take-seriously}{This website} could be helpful. I recommend: Create a new file in which you write bullet points about what you want to talk in each section. Then use the bullet points to guide you in moving the paragraphs you have already written into the correct sections.\aloff


Yosemite Valley, located in Yosemite National Park-CA, is famous for its kilometric near-vertical to overhanging walls. Most of those cliffs are constituted of resistant granitic rock \citep{bateman1992plutonism}, and although they are composed of such resistant material, weathering and erosion throughout the geological time break them down, transporting this material towards the valley in form of rockfall, rock slides, debris flow, and rock avalanches. The record of such dynamic process became well preserved in Yosemite Valley since the retreat of the last glacier, locally known as Tioga. Due to the relatively low elevation of Yosemite Valley, it is likely that this portion of the park was free of ice by 15,000 years \citep{Wieczorek+1996}. This then formed the glacial lake Yosemite, and it is possible that mass wasting deposits have been intercalating with fluvial, deltaic and lacustrine sediments from the glacial lake Yosemite stage.   In other words, the absence of moving glaciers, allied with the low gradient of the Yosemite Valley (~3m/km), created the perfect condition to preserve mass wasting deposits in the valley. 

Unlike Rockfall, which happen very frequently in the park and produce small volume of debris, rock avalanches are rare, and typically move large masses of material over long distances in a matter of seconds (Stock and Uhrhammer, 2010). Consequently, these events are an important geomorphic process reshaping the landscape of Yosemite Valley. 

  Cosmogenic \ce{^{10}Be} exposure ages demonstrate that the RAMRA happened \SI{14030 \pm 340}{yr\, B.P.} (G. Stock, personal communication). This makes this avalanche deposit the oldest dated rock avalanche in Yosemite Valley. More importantly, because the Royal Arches Meadow Rock Avalanche happened approximately at 14000 B.P., the interface between the bottom of the RAMRA and the underlying paleo-valley marks the elevation state of the valley shortly after the LGM \COMON(15,000 to 20,000 years ago)\COMOFF. 

Therefore, mapping this interface can give us insights about paleo-elevation of Yosemite Valley shortly after the LGM, and better understand \alon (``better understand'' is very vague. It's not clear what better means. Also, what does ``understand'' mean...) \aloff of the local rates of aggradation. In addition, mapping the extent of the rock avalanche may provide valuable information for future research related to the dynamics of those events, because the runout extent of an avalanche, along with its associated volume and slope, are typically used to understand the event kinematics, which can also improve risk assessment (e.g., \cite{wieczorek1998rockfall}; \cite{guzzetti2003rockfall}; \cite{stock2014quantitative}. \alon (We need to be careful not to promise something that we don't do in this paper. Future research should go into the discussion or conclusions section, not introduction.) \aloff





\section{Study Area}

Yosemite Valley, located in Yosemite National Park -- CA (Fig.~\ref{Study_Area}A), is a ~1  km deep glacially carved canyon in the Sierra Nevada, California, USA (\cite{matthes1930geologic}; \cite{huber1987geologic}). It is home to some of the largest granitic rock faces in the world \alon (need citation or remove)\aloff. Yosemite Valley is the home of at least ten rock avalanche deposits \alon(citation)\aloff. We investigated the oldest of them, the Royal Arches Meadow rock avalanche (RAMRA) (Fig.~\ref{Study_Area}B), situated in eastern Yosemite Valley. \alon(It is extremely important that all the figure labels and numbers are easily legible. See for example Austin's paper for what)\aloff

									\begin{figure*}[h]

	\includegraphics[width=\textwidth]{Figures/Study_Area.pdf}
		\caption{: (A) Map of view of Yosemite Valley - CA, red box highlights the study area (modified after Brody et al., (2015)). (B) oblique view of the study area.  \label{Study_Area}}


									\end{figure*}

The proximal portion of the study area (NE) \alon(make sure that the figure clearly shows (with writing in the figure), which one is proximal, distral, etc)\aloff, close to the talus deposits (yellow portion in (Fig.~\ref{Study_Area}B) is marked by large boulders of several meters (Fig.~\ref{Study_Area2} A and B). Moving towards the distal portion of the avalanche deposit (SW direction), we observed that the boulders decrease in size and surficial expression (from several meters to centimeters scale) (Fig.~\ref{Study_Area2} C and D), until they completely disappear in the Southernmost portion close to the Tenaya Creek cut-bank (Fig. 4 G).

\COMON
Talk about topography here

Possibky topographic profiles?
\COMOFF

\alon (I would talk about topography after the LiDAR section) \aloff

									\begin{figure*}[h]

	\includegraphics[width=\textwidth]{Figures/Study_Area2.pdf}
		\caption{: Surface morphology of the Royal Arches Meadow Rock Avalanche. Walking from the proximal portion (A-B) at NE, towards the distal portion (C-D) at SW. \label{Study_Area2}}

									\end{figure*}
			
A common strategy deployed to study, and map rock avalanche deposits is the combination of LiDAR and other topographic tools along with field observations. These techniques allow researchers to investigate surficial morphological expressions, as well as volume, and runout of rock avalanche deposits. As example, \cite{Wieczorek+1996} estimated the volume of mass wasting deposits in Yosemite Valley using  a Digital Elevation Models (DEM) of the present topography, and  another DEM representing the hypothetical datum of the flat valley level soon after filling of glacial lake Yosemite. 

In other words, to precisely estimate the volume of mass wasting deposit you would need to identify the top and the bottom of the deposit \alon(That's not true! You also need to know the location of the edges, which we couldn't find! Also, i wouldn't mention that we need to know the ``top'', as that is trivial.)\aloff. The top of the deposit can be inferred from the current topography. However, the bottom of the deposit cannot be inferred as the modern valley floor, given that Yosemite Valley has been aggrading since the LGM. This is why the second DEM is called hypothetical.


Several evidences suggests that the valley has been aggrading since the LGM: (1)The morphological surficial expressions of older deposits such as the Royal Arches Meadow rock Avalanche (\SI{\approx14}{\kilo a}) and the Glacier Point Rock Fall deposit (\SI{9.6 \pm 1}{\kilo a}) \citep{cordes2013supporting} are faded when compared to more recent rock avalanches deposits such as the El Captain Rock Avalanche (\SI{\approx3.6}{\kilo a}) \citep{stock2010catastrophic}. In addition, the work of \cite{cordes2013supporting} demonstrated that the Valley beneath the glacier point (Figure~\ref{Study_Area}) has aggraded approximately 5m since the LGM. \alon(should this entire paragraph go into introduction?)\aloff

Therefore, to study the Royal Arches Meadow Rock Avalanche and to gain insights into the elevation of Yosemite Valley shortly after the LGM, we used a combination of geophysical methods to investigate the extent of the avalanche deposit and the interface between the bottom of the rock avalanche deposit and the paleo-valley e floo. \alon This also sounds more like introduction \aloff



\section{Methods}

%Geophysical methods have been successfully applied to image mass wasting deposits (e.g., \cite{sass2006determination}; \cite{otto2006comparing}; \cite{socco2010geophysical}; \cite{brody2015near},\cite{liu2018near}), allowing researchers to precisely estimate \alon (precisely? That is not really true. Geophysical methods always see the subsurface through a biased lens. But the methods do add a dimension that can not be seen just from the surface expression of a deposit)\aloff volume, image internal structures, and to track lateral continuity in depth. \cite{doetsch2012constraining} have demonstrated the effectiveness of using Ground Penetrating Radar (GPR) to constrain electrical resistivity tomography (ERT). Inspired by their work, we combined ERT and GPR surveys to image the subsurface of the Royal Arches Meadow rock avalanche. \alon (I think the story is rather: We would like to only use GPR, because it is the easiest, fastest, most detailed. But we can't, because there are a couple of continuous reflectors that could be candidates for the bottom of the rock avalanche. So we add ERT in the mix, to discern between the units) \aloff Cosmogenic 10Be exposure ages demonstrate that the Royal Arches Meadow Rock Avalanche happened 14,030 ± 340 BP (G. Stock, personal communication).permittivity and electrical resistivity contrast between the rock avalanche and the underlying valley floor. We georeferenced the geophysical data using differential GPS measurements acquired with a Trimble Geo 7X GPS, and the elevation information from LiDAR data.



	\subsection{Ground Penetrating Radar (GPR)}
										
%%Ground Penetrating Radar is a noninvasive geophysical technique that detects electrical discontinuities in the shallow subsurface (typically < 50 m) \citep{neal2004ground}. The most common form of GPR measurements, called common offset profiling, involves keeping a transmitter antenna and a receiver antenna at a fixed distance and moving them along a profile line on the surface to detect reflections from subsurface features \citep{jol2008ground}
        \alon[I removed the intro paragraph to GPR because: 1) it is not necessary and 2) you talked about GPR before this sentence. It doesn't make sense to talk about something and then, three sections later, explain what it is]\aloff


%%Eight GPR profiles were collected with a Sensors \& Software PulseEKKO Pro (\SI{50}{\mega Hz}) system (Figure~\ref{GPR profile 7}). Note that due to obstacles in the field (e.g., boulders and fallen trunks), some of the profiles curve and do not follow a straight path (parallel or perpendicular to the possible rock avalanche flow). The profiles were then processed using GPRPy \citep{plattner2019comunity}. Processing flow and corrections included: time zero correction, filter (DEWOW and mean trace removal), T-pow gain, velocity analysis (based on WARR data), migration \citep{stolt1978migration} and topographic correction. Scripts with detailed information about the processing steps as well as the raw data for is provided for reproduction. \COMON is there a best way to phrase it? \COMOFF  

%We then used the processed profiles to track the lateral continuity of specific reflectors following the methodology proposed by \citep{mitchum1977seismic}. 

											\begin{figure*}[h]

	\includegraphics[width=\textwidth]{Figures/GPR_ERT_Map.pdf}
		\caption{of two section (G5.1 and G5.2). (B) Colored dots illustrate ERT transects at the study area (each individual dot represents an electrode position). Coordinate 0:0 is centered at zone 11S: Easting 274170 m and Northing 4180400 m.  \label{GPR profile 7}}

											\end{figure*}										
											
												
												
		\subsection{Electro Resistivity Tomography (ERT)}

%Electrical Resistivity Tomography operates by sensing the apparent resistivity of the subsurface. In the field, we insert electrodes (metal stakes) into the surface, then we plug them in a cable, and connect the cable in a battery. Subsequently, we run electrical current trough the electrodes and measure the potential difference. This then give us values of apparent resistivity. Finally, to convert the values from apparent resistivity to real resistivity, we run an inversion.
                
%\alon [again removed intro paragraph for same reason as in GPR section] \aloff

%Five ERT transects were collected at the study area (Figure 27) using the Advanced Geosciences Inc SuperSting R1 (28 electrodes), with an electrode spacing of six meters, and a Wenner–Schlumberger and dipole-dipole acquisition scheme. Subsequently, the ERT data were inverted using BERT/GIMLi (\cite{gunther2006three} and (\cite{Ruecker2017}).

	
                \subsection{Constraining ERT with GPR}

%\alon [Perhaps rather: ``Discerning (or identifying?) GPR interfaces using ERT'']
										
%GPR profiles G1, G2, G3, G4, G5 were collected following the same path of ERT profiles E1, E2, E3, E4, and E5. This strategy allowed us to verify the reliability of the data collected, not only by comparing the respective profiles but also by combining them. \alon [This narrative is quite different from the original story. Also, I would first talk about GPR, then about ERT. Because: GPR is the main data. ERT is only to help discern between reflectors seen in GPR.] \aloff Qualitatively, we compared each GPR profile with its respective ERT profile, and quantitatively, we used the GPR profiles as regularization to improve the ERT inversion. Detailed information about this technique can be found at \cite{doetsch2012constraining}.

%\COMON do we wanna a paragraph here illustrating an example how we did it with a figure?\COMOFF
%\alon [should not be necessary, as this techique is often used. Perhaps even cite some more papers, where people use this technique]\comoff									
									
									
									
									
\section{Results}
										
%Our results were obtained from the analysis of eight GPR and five ERT processed profiles. From this data we were able to identify the interface between the bottom of the Royal Arches Meadow Rock Avalanche and the palleo-valley floor. We were also able to infer the extent of the avalanche deposit.  

%In order to identify the limits between the bottom of the rock avalanche deposit and paleo-valley floor using GPR and ERT, we had to look for:

%\begin{itemize}
 %   \item In the GPR profiles: a continuous reflector in multiple directions in the hundreds of meters scale.
  %  \item In the ERT profiles: a sharp contrast between high resistivity material (rock avalanche deposit) overlaying conductive material (paleo-valley floor sediments).
%\end{itemize}

%\alon [I like this description. But I think it should go into the methods section] \aloff


%		\subsection {GPR Results}
										
%Since the rock avalanche studied here has a large surficial expression (hundreds of meters scale), we expect that its expression in the subsurface would be similar or even larger \alon [I dont understand] \aloff. Following this line of reasoning, we browsed the GPR profiles for continuous reflectors along and across profiles that could possibly represent the bottom of the rock avalanche. 

%We started the identification process at profiles G3,  G4, and G6 where the expression of four reflectors can be easily tracked along the profiles. Those four reflectors are not only present in those three profiles, but they also respectively appear in approximately the same elevation (Figure~\ref{Profiles_G3_G4_G6}). This is an indicative that those reflectors are continuous along and across the profiles. And to verify this idea, we plotted all the profiles in perspective and tracked the same four reflectors along and across profiles (Figure~\ref{Profiles_G3_G6}). This approach confirmed the lateral continuity of those four reflectors. We named them Kappa, Alpha, Betta, and Gamma. And  a quantitatively analysis reveled their elevation information\COMON Table 1. \COMOFF	


%								 \begin{figure*}[h]

%	\includegraphics[width=\textwidth]{Figures/Profiles_G3_G4_G6.pdf}
%		\caption{: (A) map view showing the position of the GPR profiles, (B) profile G4, (C) section of profile G6 and (D) profile G3. Yellow lines highlight the position along the profile of reflectors Alpha, Beta and Gamma. \label{Profiles_G3_G4_G6}}

%								   \end{figure*}
								   
								   
%								   \begin{figure*}[h]

%	\includegraphics[width=\textwidth]{Figures/Profiles_G3_G6.pdf}
%		\caption{: Three-dimensional view of profile G3 crossing profile G6. (A) continuous reflectors identified across GPR profiles. (B) Lateral continuity of reflectors Kappa, Alpha, Beta, and Gamma highlighted with yellow lines. Inferred continuity of the reflectors marked as dotted yellow line.  \label{Profiles_G3_G6}}

%								   \end{figure*}
	


	

%    \subsection {ERT Results}

%All the four reflectors identified in the GPR profiles satisfy the premise of a "continuous reflector in multiple directions in the hundreds of meters scale" as bottom of the rock avalanche deposit. Therefore, to distinguish wihch of those reflectors is indeed the bottom of the rock avalanche we looked into the ERT inversion results constrained with GPR reflectors \COMON Figure ERT + Reflectors \COMOFF. It is expected that the boundary between the rock avalanche material and the paleo-valley floor would present a sharp contrast between resistive and conductive material (respectively). And reflector Beta, at elevation \COMON 1207.06m more or less 1.1  \COMOFF marks such transition. 

										
%\section{Interpretation}

   % \subsection{Radarfaceis}
    
%Rock avalanches have a high degree of complexity in regards of movement and deposition. In addition, the facies association of such deposits still not well explored/studied by researchers \citep{otto2006comparing}. Here, we attempt to use: (1) shape of reflection; (2) dip of reflections; (3) relationship between reflectors; and (4) reflection continuity to gain insights into the architectural elements of the rock avalanche deposits and the surrounding sediments. This approach was first proposed by \cite{mitchum1977seismic} for seismic stratigraphy and latter adapted to GPR (e.g., \cite{neal2004ground}),


 %   \subsection{Extent of the Rock Avalanche}

%Unlike the bottom of the rock avalanche that has a strong and well marked reflector in the GPR profiles, the lateral extent of the deposit did not present any well marked reflector as indication of a edge (e.g., wedge shaped reflector). There three possible reasons for that: 

%\begin{enumerate}[label=\Roman*.]
  %  \item The lack of resolution (capacity of distinguish different layers) in the GPR data. In order to investigate deeper, we had to use a 50 MHz antenna, which allowed us to visualize reflectors as deep as 17 meters, but has a low resolution of XX m in this medium. 
  %  \item This i a 14Ka rock avalanche deposit, and it is possible that weathering, erosion, and deposition has smoothed the expression of the deposits edges. 
   % \item complex flow?
%\end{enumerate}


%Since no strong reflectors were identified as the laterla edges of the depost, we had to  relay on the ERT data to infer the extent of the rock avalanche deposit. But conscious that this position is possibly of by XXX meters, given the uncertainties associated with ERT inversions and data. If Beta is indeed the paleo-valley floor aprox. 14 ka, it should be continuous and trackble almost all valley long, considering that this surface was once the valley floor. Contrarily, it is likely that the RAMRA deposit covers just an specific range (hundreds of meters) equivalent to range of the resistive material (\COMON > XXXX Ohm m\COMOFF) observed in the ERT profiles. 



%Following this idea, We marked the distal (SW-nmost) limit of the rock avalanche deposit at ERT profiles E3 and E4 \COMON Figure X and Y \COMOF at the profile distance of respectively XXXm and YYYm, where the extent of resistive material at a depth of aprox. 1207m ends 
    

%For instance, we were able to track Beta in GPR profile G2, however, on ERT profile E2 (following the same path) the resistive material do not appears at the depth of 1207m until 155m along this profile. Consequently, the position 155m at ERT profile G2/E2 is likelly to mark the western edge of the RAMRA. A similar scenario was observed in profiles G1 and E1, where resistive material does not appear at the depth of 1207m until the position 225m along profile E1.

%We were not able to infer the eastern edge of the deposit. Unlike the western and the distal portion of the deposit where we could easily identify a transition between resistive and conductive material in the ERT profiles at the depth of approx 1207m, the resistive material at this portion of the study area appears all along profile E5 and it is likely to be continuous beyond it. Hence, the eastern extent of the avalanche marked on the map is an educated guess based on the geophysical observations \COMON (Figure) \COMOFF.
    
    
%    \subsection{Post Glacial Rate of Deposition?}
    
		
										
%\section{Discussion?}

										
%\section{Conclusions}

%The combination of GPR and ERT has been proven to be a powerful tool to image the subsurface of the Royal Arches Meadow rock avalanche, producing a detailed image of the local deposits. This then revealed that the interface between the bottom of the RAMRA and the palleo-valley lays on the elevation of approximately 1207m, and that the rock avalanche extends beyond the limits observed in surface. This interface marks the elevation of the valley shortly after the LGM, and can be used in the furute to understand post-glacial depositional rate in Yosemite Valley.  

%In addition, the morphology, extent and depth of this deposit indicates a massive falure from a nearby cliff. Highlighting the importance of better understanding the dynamics of events of such magnitude, mainly in a popular place such as Yosemite Valley in Yosemite National Park - CA (approx. 4 million per year).


\bibliography{mybibfile}

\end{document}
